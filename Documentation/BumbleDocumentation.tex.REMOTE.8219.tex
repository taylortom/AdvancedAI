\documentclass[a4paper,oneside]{report}

\usepackage{amsmath}
\usepackage{amssymb}
\usepackage{caption}
\usepackage[english]{babel}
\usepackage{fancyhdr} 
\usepackage{float}
\usepackage{multirow}
\usepackage[pdftex]{graphicx}
\usepackage{listings}      
\usepackage{pdfpages}
\usepackage{setspace}
\usepackage{url}
\usepackage{wrapfig}

\lstset{language=C,
numberstyle=\footnotesize,
%basicstyle=\ttfamily\footnotesize,
basicstyle=\footnotesize,
numbers=left,
stepnumber=1,
frame= lines,
breaklines=true}

\makeatletter


%
% Some custom definitions
%

% add horizontal lines
\newcommand{\HRule}{\rule{\linewidth}{0.5mm}}
\newcommand{\HRuleLight}{\rule{\linewidth}{0.1mm}}

% custom part page
\def\part#1#2
{
	\par\break
  	\addcontentsline{toc}{part}{#1}
	\noindent
	\null	
	\HRuleLight\\[0.0cm]
	\vspace{20pt}	 
	\begin{flushright} 		
  	{\Huge \bfseries \noindent #1}\\
  	\vspace{30pt} 
	\begin{minipage}{0.85\textwidth}
		\begin{flushright}
		{\large \noindent #2}
		\end{flushright}
	\end{minipage}\\[0.75cm] 
	\end{flushright} 		
	\thispagestyle{empty}
	\break
}

% chapter header
\renewcommand{\@makechapterhead}[1]
{\vspace*{50\p@}{
	\parindent \z@ \raggedright \normalfont
	%\huge \bfseries \thechapter. #1
	\huge \bfseries #1
	\vspace{20pt}}}

\setcounter{secnumdepth}{-1} 
\onehalfspace
\oddsidemargin 1in 
\oddsidemargin 0.6in 
\topmargin -0.3in
\setlength{\textwidth}{14cm}
\setlength{\textheight}{23cm}
\lstset{language=C} 

\begin{document}

%
% Cover page
%
\begin{titlepage}
\begin{center}

\Huge Donald Nute's VWorld\\ 
\huge \emph{Bumble 2.0}

\HRuleLight\\[0.5cm]

\begin{minipage}{0.45\textwidth}
	\begin{flushleft}\large
		\emph{Author:}\\
			\textbf{James P. \textsc{Spencer}}\\[0.27cm]
			Computer Science (Games)
			Student Number: 08809130
	\end{flushleft}
\end{minipage}
\begin{minipage}{0.45\textwidth}
	\begin{flushright}\large
		\emph{Author:}\\
			\textbf{Thomas J. \textsc{Taylor}}\\[0.27cm]
			Computer Science (Games)
			Student Number: 08813043
	\end{flushright}
\end{minipage}\\[0.75cm] 

\HRuleLight\\[0.2cm]

\large School of Computing, Engineering and Mathematics\\ \textbf{University of Brighton}

\vfill
\huge Documentation\\
\large April, 2012\\

\end{center}
\end{titlepage}

% reset page count
\setcounter{page}{1}


\section{Project Aim}

You will be working on Donald Nute's V-World throughout the majority of the workshop programme, and will have soon completed workshop exercises designed to experiment with the world, its actors and its main (adventurer) agent "Bumble."
This assignment is simple and extremely open-ended:

\begin{quotation} Modify aspects of the program to make the behaviour of the actors (including Bumble) more 'interesting' than it is at the end of the formal workshops \end{quotation}

\section{Ideas}

Interesting behaviour could be characterised in many ways, but the following is indicative:

\begin{itemize}
\item The (demonstrated) ability of Bumble to deliberate about its situation
\item The (demonstrated) ability of Bumble to react to changes in the world
\item The ability of Bumble to demonstrate intelligent decisions about actions
\item Actors in the world to demonstrate an enhanced level of intelligent behaviouretc.
\item Proof of the success of your modifications could be:
\item Behaviour based on a given set of scenarios - comparison with Bumble 1.1 \& 1.2 say
\item Performance measured against V-World's number of moves before (RIP)
etc.
\end{itemize}

\section{Deliverables}

\textbf{To be submitted}: Thursday 10 May, 2012 by 8:45am\\

\noindent \textbf{(a) The program} - submission details to be arranged\\
\textbf{(b) Documentation} - submitted both in written \& electronic form

\emph{This should be structured in the normal way for such documents, e.g.  introduction, strategy, modifications made (with relevant code snippets), test plan, outcomes, further work, conclusions. It should support the program and be around 3000 words}

\section{Marking Criteria}

\begin{itemize}
\item The system is functional according to the specifications, i.e. new behaviour demonstrated - it works: 20\%
\item Level of sophistication in the new behaviour(s): 30\%
\item The code is well-structured, documented and annotated (it is clear and accessible): 10\%
\item The documentation as specified: 40\%
\end{itemize}





\section{Introduction}

\section{Strategy}

\section{Modifications}

	\subsection{Search}
	
		Search is a major area of game AI as it helps add realism and effeciency to agents movements.

In VWorld if bumble is to survice an negate the world effectivly it is vital for him to have some form of search so he can navigate back to an object he has seen for instance if he is hungry he should negate to a tree or any food he has seen. 

We have experimented with implemeing several types of search, this has been done through carrying out the activities outlined in workshops 4 and 5. 

Our inital investigations explored the use of breadth and depth first search, both of these types of search arn't well suited for our gamewold and games in general, the main resoning for this is their extreamly high worst case perfomance and the fact that they don't produce natural  or effectient search paths. 

It is for this reason we explored the use of heuristic based search, where we looked at greedy and A* search although it should be noted we did consider several other search methods that could give naturalistic approaches such as Theta* however it was demed implementing this would be nontrivial and of little value due to the grid based system present in vworld. 

Overall Implemening search has proven to be challenging and for reason we have been unable to implement A* instead sticking with greedy search.  At the highest level our search is functioning by looking for a parth to an object given a current condition, calling the search predicate will generate a path to the desired object if it is contained in bubmles map of the current world.  There are however issues that can arise in this proccess, the first issue arises when another agent such as a hornet blocks bumble's next step, in this situation he will have to avoid the desired cell and move into one adjacent to it.  This has been handled by defining a list of predicates for adjacent cells one such example of this is given our desired cell north west and it is blocked by a hornet bumble will move to either north or west if he can, afterwhich he will ease his current path.  If bumble is unable to follow the specified path and move to an adjavent cell he will erase his current path. Essentially this is a simple form of collision avoidance this could be extended furthur through the use of ray casting to achive an effect similar to that of open steer. 

Our implementation of greedy search is based off the code provided for the workshop we have had several issues with this code and for this reason we have made some minor adjustments. The main change we have implented is a counter on the amount of times the search can recuse, this is considered a hack however we have deemed this to be nessacery to prevent the heap from overflowing and crashing the application. We had tierlessly attempted to find and debug the underlying cause of this issue. Although this bug does not fully prevent search from functioning, it will however mean that the application is unlikely to run for a great deal of time as bubmle will often become stuck and die of starvation.  Overall this bug is primary barrier to hightend success of the system and in resolving it we feel that we would have had a far more impressive submission. 

	

\section{Test Plan}


\section{Evaulation}

	We have utilized the set of test maps supplied with bumble to evaulate the effectivness of our solution. In this experimentation we work with the assumption that if bumble survives more than 5000 moves he will live infinitely.
	
	???3 or 5 tests ? 
	
	
	\begin{table}[position specifier]
		\centering
		\begin{tabular}{| l | c | c | c | r |}
		\hline
			Map & Test 1 & Test 2 & Test 3 & Mean \\ \hline
			Test 1 &  X & Y & Z & X+Y+Z/3 \\ \hline
		\end{tabular}
		\caption{This table shows some data}
		\label{tab:myfirsttable}
	\end{table}
	
	
\section{Outcomes}

\section{Conclusions}

\section{Further Work}

	\subsection{Machine Learning}
		We have both compleated final year projects in this area, and this is something of great intrest, we feel that this is very intresting area of AI...

		
		
		Q-learning.
		
		Reinfocement Learning.
		
		Genetic Algorithms. 
	
	\subsection{Search Improvements}
		At current the search in the system is simple in nature, it was desired to have implemented A* but this has unfortuantly proven to be problematic.
		
		The current search could be improved by searching for the same goal once a search is erased .
		
		
		Collision avoidance.
		
		Naturalistic search. 
		
	\subsection{Mapping Improvements}
		The mapping algorithm is simple in nature and could be improved by the use of? 
		
		Boundry Detection (IE) can i see a wall ? is the cell ajcent a wall ? therefore i do not need to visit that cell because it won't reveal anymore information to me. 
	
		The problem of mapping is strongly linked to Maze exploration, there are several intresting maze exploration algorithms such as (Azkaban algorithm, Dead-end filling, Wall follower ) and we feel that the use of one of these would make the bubmles discovery of areas more effectively. 
	
		Out of these of  we feel that the wall follower algorithm is well suited to bumbles world would.... 
		
		Talk about wall follower.

\appendix
\onehalfspace

\newpage
\section{Appendix X: jfkdsgkjfdksljgklfdlkgjlkfd}


\end{document}
