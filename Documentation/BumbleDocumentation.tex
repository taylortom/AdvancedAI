\documentclass[a4paper,oneside]{report}

\usepackage{amsmath}
\usepackage{amssymb}
\usepackage{caption}
\usepackage[english]{babel}
\usepackage{fancyhdr} 
\usepackage{float}
\usepackage{multirow}
\usepackage[pdftex]{graphicx}
\usepackage{listings}      
\usepackage{pdfpages}
\usepackage{setspace}
\usepackage{url}
\usepackage{wrapfig}

\lstset{language=C,
numberstyle=\footnotesize,
%basicstyle=\ttfamily\footnotesize,
basicstyle=\footnotesize,
numbers=left,
stepnumber=1,
frame= lines,
breaklines=true}

\makeatletter


%
% Some custom definitions
%

% add horizontal lines
\newcommand{\HRule}{\rule{\linewidth}{0.5mm}}
\newcommand{\HRuleLight}{\rule{\linewidth}{0.1mm}}

% custom part page
\def\part#1#2
{
	\par\break
  	\addcontentsline{toc}{part}{#1}
	\noindent
	\null	
	\HRuleLight\\[0.0cm]
	\vspace{20pt}	 
	\begin{flushright} 		
  	{\Huge \bfseries \noindent #1}\\
  	\vspace{30pt} 
	\begin{minipage}{0.85\textwidth}
		\begin{flushright}
		{\large \noindent #2}
		\end{flushright}
	\end{minipage}\\[0.75cm] 
	\end{flushright} 		
	\thispagestyle{empty}
	\break
}

% chapter header
\renewcommand{\@makechapterhead}[1]
{\vspace*{50\p@}{
	\parindent \z@ \raggedright \normalfont
	%\huge \bfseries \thechapter. #1
	\huge \bfseries #1
	\vspace{20pt}}}

\setcounter{secnumdepth}{-1} 
\onehalfspace
\oddsidemargin 1in 
\oddsidemargin 0.6in 
\topmargin -0.3in
\setlength{\textwidth}{14cm}
\setlength{\textheight}{23cm}
\lstset{language=C} 

\begin{document}

%
% Cover page
%
\begin{titlepage}
\begin{center}

\Huge Donald Nute's VWorld\\ 
\huge \emph{Bumble 2.0}

\HRuleLight\\[0.5cm]

\begin{minipage}{0.45\textwidth}
	\begin{flushleft}\large
		\emph{Author:}\\
			\textbf{James P. \textsc{Spencer}}\\[0.27cm]
			Computer Science (Games)
			Student Number: 08809130
	\end{flushleft}
\end{minipage}
\begin{minipage}{0.45\textwidth}
	\begin{flushright}\large
		\emph{Author:}\\
			\textbf{Thomas J. \textsc{Taylor}}\\[0.27cm]
			Computer Science (Games)
			Student Number: 08813043
	\end{flushright}
\end{minipage}\\[0.75cm] 

\HRuleLight\\[0.2cm]

\large School of Computing, Engineering and Mathematics\\ \textbf{University of Brighton}

\vfill
\huge Documentation\\
\large April, 2012\\

\end{center}
\end{titlepage}

% reset page count
\setcounter{page}{1}

\section{Introduction}

At the beginning of the module, we were presented with Donald Knute's VWorld: a simple 2D sandbox game environment written in the Prolog logic programming language.

The game world itself is represented as a 2D grid structure, and can consist of a number of different levels (or rooms) interconnected with doors. Each level is filled with a number of different objects, ranging from impassable walls, to collectibles such as health power ups (fruit) and weapons (a sword), to NPCs such as snails and hornets, which try to hinder the player's progress by inflicting damage or blocking their path. 

The VWorld game framework we were given was already set up to deal with the more low-level functionality such as the user-interface elements and graphical display and loading and initialising the game. A number of different behaviours were also predefined for the non-player characters (NPCs). For example, the NPCs were able to move around the level (and guard objects in the case of Dragons and witches). As mentioned, the snails follow the player in an attempt to block their path, and the hornets attack the player, inflicting damage. 

The main protagonist of the game is Bumble: an AI character who finds himself in the mysterious realm of VWorld. He must have his wits about him at all times in order to survive this treacherous land. Bumble is given a number of tools to aid him in his fight against evil, such as a sword, a shield and bugspray. These items are scattered about the level, and must be searched down before they can be used.

\section{Aims and Objectives}

We were given a lot of freedom with this project in terms of the actual requirements. As there is no real `aim' in the VWorld game, we were left to our own devices to develop the system as we saw fit. The guidelines for the project were appropriately open-ended:

\begin{quotation} 
Modify aspects of [\emph{VWorld}] to make the behaviour of the actors (including Bumble) more 'interesting' than it is at the end of the formal workshops 
\end{quotation}

To give us some idea as to which areas we could develop, we were given a number of test level maps, which are designed in such a way as to encourage certain behaviour. For example, some of the test levels included the 'bird' character who must be caught to complete the level. Catching the bird requires Bumble to collect 'birdseed' located somewhere in the level, and so it can easily be seen that some sort of objectives are required for Bumble to succeed in such levels. Similarly, some levels have locked doors which require specific keys to enter.

There were a number of different angles from which we could approach this problem. Our first goal with this project was to simply increase Bumble's survival rate, as we felt that this was needed before any other aspects of the game could be improved. As a part of this objective, we also wanted to fix a number of issues with the basic Bumble and VWorld framework. Moving on from these small incremental changes, we wanted to implement some sort of search algorithm which Bumble could use in combination with his dynamic map of the world to find objects he had previously seen.

We used the test levels to create a basic 'specification' of the kind of behaviour we wanted Bumble to exhibit:

\begin{itemize}
	\item Some form of search
	\item Refine Bumble's strength/damage system
	\item Implement some exploration
\end{itemize}

Our overall goal was to iterate rather than to innovate, by developing the existing behaviour, fixing existing issues, and by implementing more incremental changes.

\section{Basic Bumble}

To start us off, we were given a very basic Bumble which we could use as a starting point for our development. The original Bumble was able to randomly explore the environment, but little else. In the tutorial sessions, we began to develop some basic behaviour for Bumble. By the end of the tutorials, Bumble was able to:

\begin{itemize}
	\item Randomly navigate the world
	\item Pick up any interesting/desirable objects he was stood next to
\end{itemize}





By the end of the tutorials, Bumble was able to move around the game world, and build a 'map' of where certain objects were located as they were found. 


Bumble has a few basic stats which are updated as he roams the level: strength, which decreases as Bumble moves around the level and can be replenished with tasty apples, and damage, which is incremented if Bumble is attacked. Bumble dies if either his strength reaches 0, which means he is unable to move, or if his damage reaches 100\%.

Bumble also had a few basic rules to help with his survival in the dangerous world. These rules are used to ascertain whether Bumble is hungry or hurt based on his current strength and damage levels. This information is then used to point Bumble in the direction of any power-ups as appropriate. This behaviour was quite naive, as Bumble would only move towards power-ups in his immediate vicinity (i.e. less than 2 squares away). In addition to this, Bumble also collected any 'interesting' or 'desirable' objects that he was standing next to.

As it was, Bumble's behaviour was still very naive. Bumble only considered objects which were very close by (i.e. within a two square 'radius'). Although we had a map of the world which Bumble created dynamically as he explored the world, nothing was done with this map. The way Bumble treated objects in the game was also very simplistic: 'interesting' or 'desirable' objects were only picked up if Bumble was stood directly beside them, and Bumble had no real decision-making process with regards to picking up fruit or crosses. As it stood, Bumble would simple become 'hungry' when his strength dropped to a certain level, and 'hurt' when his damage exceeded a set value; there was no reasoning behind his decisions. Additionally, 

; he was only able to react to objects he was close to, despite the fact that a map was dynamically generated as Bumble explored the level. There was also no real 'method' behind Bumble's navigation, he simple wandered around randomly. Bumble also had no sense of any 'objectives' in the level. For example, he had no idea that in order to catch the bird, he first needed some birdseed. There were also a number of small bugs in the existing system which hindered his progress. 

\section{Modifications}

One of the key elements in our strategy was the smaller, more incremental changes and bug-fixes to the original system.


There were a lot of small modifications which were made to improve the basic version of Bumble.

The hornets follow Bumble upon spying him.

Bumble no longer flees if in possession of the shield/sword

When Bumble becomes hurt, he checks the map for the appropriate resources to fix him.
Fixed a bug whereby Bumble could get stuck alternating between two interesting objects (e.g. tree/cross)

Hornets no longer cause injury if Bumble's in possession of the bugspray. Also, the bugspray now actually works.

Added a conversion from cartesian coordinates to directions
Bumble now spies desirable/interesting objects from 2 squares away
Fixed a bug whereby Bumble could get stuck in a corridor 1 sq wide (see castle2.vw). 
If there was an interesting/ desirable object at the end of the corridor, Bumble would try to go west, but would simple stay in the same sqaure. There's now a check to see if Bumble already attempted the same move, if so, he moves randomly.

A map of the entire level is built. When Bumble goes through a door, an extra square is added in the appropriate direction to account for the wall in-between the levels. 

As removable items are taken by Bumble, they are also removed from the map.

Implemented basic fuzzy reasoning for the strength/health. Bumble now has more levels: 
Not hungry, small hungry, medium hungry, starving etc. These levels are used when determining whether Bumble should consume fruit/use crosses. If Bumble sees fruit, but is very hurt, it will pass by and continue looking for a cross.

Added an exploration mode which is enabled/disabled based on Bumble's current health/strength. If not hungry/hurt, Bumble will explore the map for areas not visited using the preference: e, nw, se, sw, n, e, s, w. As Bumble explores, the squares he visits are marked off, removing them from the exploration mode. Bumble stops exploring if he can't see any nearby unvisited squares.

\section{New Features}

\subsection{Search}

Some random words...

\subsection{Fuzzy-Inspired Reasoning}
	
Didn't bother with the probability component, as it caused unpredictable and unrealistic behaviour.

\section{Team Development}
	
\section{Testing and Results}

We have utilised the set of test maps supplied with bumble to evaluate the effectiveness of our solution. In this experimentation we work with the assumption that if bumble survives more than 5000 moves he will live infinitely.
	
	???3 or 5 tests ? 

\begin{table}[h!]
	\centering
	\begin{tabular}{| l | c | c | c | r |}
	\hline
		Map & Test 1 & Test 2 & Test 3 & Mean \\ \hline
		Test 1 &  X & Y & Z & X+Y+Z/3 \\ \hline
	\end{tabular}
	\caption{This table shows some data}
	\label{tab:myfirsttable}
\end{table}

\section{Analysis}

What we found out from the results

\section{Conclusions}

\section{Further Work}

\subsection{Machine Learning}
We have both completed final year projects in this area, and this is something of great interest, we feel that this is very interesting area of AI...

		
		
\paragraph{Q-learning}
		
\paragraph{Reinforcement Learning}
		
\paragraph{Genetic Algorithms}
	
\subsection{Search Improvements}
		At current the search in the system is simple in nature, it was desired to have implemented A* but this has unfortunately proven to be problematic.
		
		The current search could be improved by searching for the same goal once a search is erased .
		
		
		Collision avoidance.
		
		Naturalistic search. 
		
	\subsection{Mapping Improvements}
		The mapping algorithm is simple in nature and could be improved by the use of? 
		
		Boundary Detection (IE) can i see a wall ? is the cell adjacent a wall ? therefore i do not need to visit that cell because it won't reveal anymore information to me. 
	
		The problem of mapping is strongly linked to Maze exploration, there are several interesting maze exploration algorithms such as (Azkaban algorithm, Dead-end filling, Wall follower ) and we feel that the use of one of these would make the Bumble's discovery of areas more effectively. 
	
		Out of these of  we feel that the wall follower algorithm is well suited to bumbles world would.... 
		
		Talk about wall follower.

\end{document}
